\documentclass[oneside,a4paper]{article}
% \documentclass[twoside,a4paper]{article}

% 编码相关.
\usepackage[utf8]{inputenc} % 字符编码.
\usepackage{ctex} % 支持中文.
\usepackage{color, xcolor} % 颜色.

% 目录标题引用.
\usepackage{abstract} % 摘要.
\usepackage[hidelinks]{hyperref} % 超链接[colorlinks, linkcolor=red, anchorcolor=blue, citecolor=green]
\usepackage{authblk} % 使用\thanks定义通讯作者, 以及脚注的内容.
\usepackage{titlesec} % 修改章节编号.
\usepackage{cite} % 参考文献.
\usepackage{appendix} % 附录.

% 排版相关.
\usepackage{multicol} % 多列排版.
\usepackage{fancyhdr} % 页眉页脚.
\usepackage{listings} % 代码环境.
\usepackage{graphicx, subfig} % 插图.
\usepackage{float} % 浮动位置.
\usepackage{enumerate} % 编号itemize.

% 数学环境.
\usepackage{amsfonts} % mathbb, mathfrak.
\usepackage{eucal} % mathcal命令是欧拉书写体, mathscr.
\usepackage{amsmath} % 定义多行公式环境和一系列排版数学公式的命令, 例如连分数\cfrac.
\usepackage{amssymb} % 它定义了amsfonts宏包里msam和mabm字库中全部数学符号的命令.
\usepackage{gensymb} % 数学符号补充.
\usepackage{amsthm} % 定义proof环境, 能自动添加证毕符号, \newtheorem{定理环境名}{标题}[计数器名].
\usepackage[ruled]{algorithm2e} % 算法环境包
\usepackage{tikz} % 绘图.
\usetikzlibrary{positioning}

% 页面参数.
\usepackage{geometry}
% \geometry{a4paper,left=2cm,right=2cm,top=1cm,bottom=1cm}
% \geometry{margin=1.5cm, vmargin={0pt,1cm}}
% \geometry{hmargin={3.18cm, 3.18cm}, width=14.64cm, vmargin={2.54cm, 2.54cm}, height=24.62cm}
% \setlength{\topmargin}{-1cm}
% \setlength{\paperheight}{29.7cm}
% \setlength{\textheight}{25.3cm}
% \setlength{\headheight}{15pt}

% \pagestyle{empty} % 没有页眉和页脚
\pagestyle{fancy} % 一般格式
% \pagestyle{plain} % 没有页眉, 页脚中部放置页码.
% \pagestyle{headings} % 没有页脚, 页眉包含章节的标题和页码.
% \pagestyle{myheadings} % 没有页脚, 页眉页码和使用者所定义的信息.
% \fancyhead{}
% \lhead{lhead}
% \chead{chead}
% \rhead{rhead}

\setcounter{secnumdepth}{3}
\setcounter{tocdepth}{2}

\title{标题}
\author{姓名 \quad 学号 \quad 专业}
\date{\today}
